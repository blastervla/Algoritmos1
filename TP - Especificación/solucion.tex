\documentclass[a4paper]{article}
\input{Algo1Macros}

\usepackage{a4wide}
\usepackage{amsmath, amscd, amssymb, amsthm, latexsym}
\usepackage[spanish,activeacute]{babel}
\usepackage{enumerate}

\setlength{\parskip}{0.1em}
\usepackage{caratula} % Version modificada para usar las macros de algo1 de ~> https://github.com/bcardiff/dc-tex

\begin{document}

\titulo{TP de Especificación}
\subtitulo{Juego de la vida toroidal}
\fecha{\today}
\materia{Algoritmos y Estructuras de Datos I}
\grupo{Grupo: Java the Hutt$;$}

\newcommand{\senial}{\textit{se\~nal}}

% Pongan cuantos integrantes quieran
\integrante{Pomsztein, Vladimir}{364/18}{blastervla@gmail.com}
\integrante{Zinik, Luciano}{290/17}{lzinik@gmail.com}

\maketitle

\section{Problemas}

% 1. Dado un toroide verifica si es válido.
\begin{proc}{esValido}{\In t: $toroide$, \Out result: $\bool$}{}
    \pre{\True}
    % POST: - Tiene la misma cantidad de columnas en cada fila
    %       - No es vacía
    \post{$result = \True $ \Iff esToroideValido(t)}
\end{proc}

% 2. Dado un toroide devuelva todas las posiciones vivas.
\begin{proc}{posicionesVivas}{\In t: $toroide$, \Out vivas: $\TLista{\ent \times \ent}$}{}
    \pre{esToroideValido(t)}
    % La posición pertenece a result sí y sólo sí está viva
    \post{(\forall i, j : \ent)(enRango(t, i, j) \implicaLuego ((i, j) \in vivas \Iff estaViva(t[i][j])))}
\end{proc}

% 3. Dado un toroide devuelva su densidad de población, es decir, la relación entre 
% la cantidad de posiciones vivas y la cantidad total de posiciones
\begin{proc}{densidadPoblacion}{\In t: $toroide$, \Out result: $\float$}{}
    \pre{esToroideValido(t)}
    % densidad = cantidad vivas / cantidad total
    \post{result = cantidadVivas(t) / cantidadTotal(t)}
    \aux{cantidadTotal}{t: $toroide$}{\ent}{
        filas(t) \times columnas(t)
    }
\end{proc}

% 4. Dado un toroide y una posición del mismo, indique si dicha posición estaría viva 
% luego de un tick.
\begin{proc}{evolucionDePosicion}{\In t: $toroide$, \In posicion: $\ent \times \ent$, \Out result: $\bool$}{}
    % la posicion tiene que estar en rango
    \pre{esToroideValido(t) \yLuego enRango(t, posicion_0, posicion_1)}
    % Evolución: 
    %   - Viva con < 2 adyacentes vivas --> muere
    %   - Viva con 2 o 3 adyacentes vivas --> vive
    %   - Viva con > 3 adyacentes vivas --> muere
    %   - Muerta con 3 adyacentes vivas --> vive
    \post{result = valorLuegoDeEvolucion(t, posicion)} % estaViva y tiene 2 o 3 adyacentes o ¬estaViva y tiene 3 adyacentes
    
\end{proc}

% 5. Dado un toroide lo evoluciona un tick.
\begin{proc}{evolucionToroide}{\Inout t: $toroide$}{}
    \pre{esToroideValido(t) \yLuego t = T0}
    % Quizás usando una sumatoria que recorra todo el toroide (actually, son 2)
    % Y que en cada término de la sumatoria cambie el estado de esa posicion
    % Alternativa (me gusta más): Definimios la metavariable t1, t = t2 sí y sólo sí cada 
    % posición de t2 es igual a dicha posición de t1 pero evolucionada
    \post{esEvolucion(T0, t)}
\end{proc}

% 6. Dado un toroide t y un natural k, devuelve el toroide resultante de evolucionar 
% t por k ticks
\begin{proc}{evolucionMultiple}{\In t: $toroide$, \In k: $\ent$, \Out result: $toroide$}{}
    % k debe ser natural
    \pre{esToroideValido(t) \land k \geq 0}
    % result = t1 <=> LA SUMATORIA da 0
    % LA SUMATORIA da 0 <=> la evolución de t = evolucion de t1 (se puede porque
    % evolucionToroide es inout, si es distinta suma 1)

    % se puede definir un aux recursivo que checkee la evolución por ticks hasta que sea 0
    % y que devuelva un booleano si es igual
    \post{ esEvolucionMultiple(t, result, k) }
\end{proc}

% 7. Dado un toroide devuelve si el mismo es periódico o no. En caso de serlo, 
% se debe devolver en p la mínima cantidad de ticks en la cual se repite el patrón. 
% Decimos que un toroide es periódico si pasada cierta cantidad de ticks, vuelve a tener
% exactamente la misma configuración que tenía originalmente.
\begin{proc}{esPeriodico}{\In t: $toroide$, \Inout p: $\ent$, \Out result: $\bool$}{}
    \pre{esToroideValido(t)}
    \post{(result = \True) \Iff (\exists ks: \TLista{\ent})((\forall i: \ent)((0 < i < length(ks))\implicaLuego((ks[i - 1] > 0) \yLuego (ks[i - 1] < ks[i]))\yLuego p = ks[0]}
\end{proc}

% 8. Dados dos toroides, devuelve si uno es la evolución múltiple del otro.
\begin{proc}{primosLejanos}{\In t1: $toroide$, \In t2: $toroide$, \Out primos: $\bool$}{}
    \pre{esToroideValido(t1) \land esToroideValido(t2)}
    \post{primos = \True \Iff 
    ((\exists k: \ent)((k > 0)\yLuego(
        (esEvolucionMultiple(t1, t2, k))
        \oLuego
        (esEvolucionMultiple(t2, t1, k))
        )))}
\end{proc}

% 9. Dada una secuencia de toroides, devuelve el índice de aquel toroide que más ticks 
% tardará en morir. Se considera que un toroide muere cuando no tiene posiciones vivas.
\begin{proc}{seleccionNatural}{\In ts: $\TLista{toroide}$, \Out res: $\ent$}{}
    \pre{todosToroidesValidos(ts) \yLuego algunToroideMuere(ts)}

    \post{(\exists is: \TLista{\ent})((\forall i: \ent)(((i \in is)\yLuego(0 < i < length(is)))\\\implicaLuego(cantidadDeTicksHastaMuerte(ts[i - 1]) \geq cantidadDeTicksHastaMuerte(ts[i])))\yLuego res = is[0])}
    \pred{todosToroidesValidos}{ts: $\TLista{toroide}$}{
        (\forall t: $toroide$)((t \in ts) \implicaLuego esToroideValido(t))
    }
    \pred{algunToroideMuere}{ts: $\TLista{toroide}$}{
        (\exists t: $toroide$)((t \in ts)\yLuego(muere(t)))
    }

    \aux{cantidadDeTicksHastaMuerte}{t: $toroide$}{\ent} {
        \\\IfThenElse{(\exists tx: $toroide$)((\forall x \in tx)\implicaLuego \neg estaViva(x))\yLuego(\exists is: \TLista{\ent})((\forall i : \ent)((0 < i < length(is))\implicaLuego((is[i - 1] < is[i]) \yLuego esEvolucionMultiple(t, tx, is[i - 1])))))}{
            is[0]
        \\}{
            -1
        }
    }

    \pred{muere}{t: $toroide$}{
        cantidadDeTicksHastaMuerte(t) \neq -1
    }
\end{proc}

% 10. Dados dos toroides de la misma dimensión, devuelva otro (de la misma dimensión) 
% que tenga vivas solo aquellas posiciones que estaban vivas en ambos toroides
\begin{proc}{fusionar}{\In t1: $toroide$, \In t2: $toroide$, \Out res: $toroide$}{}
    \pre{(esToroideValido(t1) \land esToroideValido(t2)) \yLuego (filas(t1) = filas(t2) \yLuego columnas(t1) = columnas(t2))}
    % res = tf <=> para todo i,j estaViva(tf[i][j]) 
    % <=> estaViva(t1[i][j]) y estaViva(t2[i][j])
    \post{res = tf \Iff (\exists tf: toroide)((filas(tf) = filas(t1) \yLuego columnas(tf) = columnas(t1))\yLuego(\forall i, j: \ent)((enRango(tf, i, j))\implicaLuego(estaViva(tf, i, j) \Iff estaViva(t1, i, j) \yLuego estaViva(t2, i, j))))}
\end{proc}

% 11. Dados dos toroides de la misma dimensión, indica si uno es el resultado de trasladar
% la vista en el otro. Es decir, que moviendo el centro del eje de coordenadas de uno 
% de los toroides en alguna dirección, se obtiene el otro.
\begin{proc}{vistaTrasladada}{\In t1: $toroide$, \In t2: $toroide$, \Out res: $\bool$}{}
    \pre{(esToroideValido(t1) \land esToroideValido(t2)) \yLuego mismaDimension(t1, t2)}
    \post{
        \\res = \True \Iff esVistaTrasladada(t1, t2)}
\end{proc}

% 12. Verifica si la menor superficie que cubre a todas las celdas vivas se va incrementar 
% en el próximo tick
\begin{proc}{enCrecimiento}{\In t: $toroide$, \Out res: $\bool$}{}
    \pre{esToroideValido(t)}
    \post{res = \True \Iff 
    \\(\exists te: $toroide$)(esEvolucion(t, te)\yLuego crecio(t, te))}

    \pred{crecio}{t: $toroide$, te: $toroide$}{
        (\exists s, se: \ent)\\((esMenorSuperficie(t, s) \land esMenorSuperficie(te, se))\yLuego s < se)
    }

    \pred{esMenorSuperficie}{t: $toroide$, s: \ent}{
        \\(\exists ts: \TLista{$toroide$})((compuestoPorVistasTrasladadas(ts, t) \y ordenadoSuperficieAscendente(ts)) \\\yLuego s = ts[0])
    }
    \pred{compuestoPorVistasTrasladadas}{ts: \TLista{$toroide$}, t: $toroide$}{
        (\forall tx: $toroide$)\\((tx \in ts)\implicaLuego esVistaTrasladada(tx, t))
    }
    \pred{ordenadoSuperficieAscendente}{ts: \TLista{$toroide$}}{
        (\forall i: \ent)((0 < i < length(ts))\\\implicaLuego(esSupMayorOIgual(ts[i], ts[i - 1]))))
    }
    \pred{esSupMayorOIgual}{t1: $toroide$, t2: $toroide$}{
        (\exists s1, s2: \ent)\\((esSuperficie(t1, s1) \land esSuperficie(t2, s2))\yLuego s1 \geq s2)
    }
    \pred{esSuperficie}{t: $toroide$}{
        % Tomar bolsa de superficies posibles y tomar la menor que tenga la misma cantidad
        % de celdas vivas que nuestro toroide
        % Tendríamos (xStart, yStart), (xEnd, yEnd) : Z x Z, de manera que contando las celdas
        % vivas desde xStart a xEnd columnas en cada fila desde yStart a yEnd tiene la 
        % misma cantidad que nuestro toroide original
        (\exists rs: \TLista{(\ent \times \ent)\times(\ent \times \ent)})
        \\(compuestoPorEquivalentes(rs, t)\yLuego((\forall i: \ent)((0 < i < length(rs))\implicaLuego
        \\((0 \leq rs[i]_{0_0} \leq rs[i]_{1_0} \leq columnas(t))\land(0 \leq rs[i]_{0_1} \leq rs[i]_{1_1} \leq filas(t)))
        \\\yLuego(area(rs[i]) \geq area(rs[i - 1])))))
    }
    \pred{compuestoPorRangosEquivalentes}{rs: $\TLista{(\ent \times \ent)\times(\ent \times \ent)}$, t: $toroide$}{
        (\forall rect: (\ent \times \ent)\times(\ent \times \ent))((rect \in rs)\implicaLuego (contarVivasEnArea(t, rect) = cantidadVivas(t)))
    }
    $/* rect = (xStart, yStart) \times (xEnd, yEnd) */$\\
    \aux{contarVivasEnArea}{t: $toroide$, rect: $(\ent \times \ent)\times(\ent \times \ent)$}{\ent}{
        \\\displaystyle \sum_{i = rect_{0_1}}^{rect_{1_1}}(\sum_{j = rect_{0_0}}^{rect_{1_0}} \IfThenElse{estaViva(t[i][j])}{1}{0})
    }
    \aux{area}{rect: $(\ent \times \ent)\times(\ent \times \ent)$}{\ent}{
        \\/* base \times altura */
        \\(rect_{1_0} - rect_{0_0}) \times (rect_{1_1} - rect_{0_1})
    }
\end{proc}

\section{Predicados y Auxiliares generales}

% === Ejercicio 1 ===
    \pred{noEsVacia}{t: $toroide$}{
        $(length(t) \textgreater 0)$ \yLuego (\forall x: \TLista{\bool})((x \in t) \implicaLuego ($length(x) \textgreater 0$))
        }
    \pred{esMatriz}{t: $toroide$}{
    (\forall x, y: \TLista{\bool})((x, y \in t) \implicaLuego ($length(x) = length(y)$))
    }
    \pred{esToroideValido}{t: $toroide$}{
        (noEsVacia(t) \land esMatriz(t))
    }
    
    \pred{filas}{t: $toroide$}{
        $length(t)$
    }
    \pred{columnas}{t: $toroide$}{
        \IfThenElse{$filas(t) \textgreater 0$}{$length(t[0])$}{0}
    }
% ===================
% === Ejercicio 2 ===
    \pred{estaViva}{x: \bool}{
        x = \True
    }
    \pred{enRango}{t: $toroide$, i: \ent, j: \ent}{
        \IfThenElse{(0 \leq i < filas(t)) \yLuego (0 \leq j < columnas(t))}{\True}{\False}
    }

%====================

% === Ejercicio 3 ===
\aux{cantidadVivas}{t: $toroide$}{\ent}{
        \displaystyle \sum_{i=0}^{filas(t)-1}(\sum_{j=0}^{columnas(t)-1} \IfThenElse{estaViva(t[i][j])}{1}{0})
    }

% ===================
% === Ejercicio 4 ===
    \aux{valorLuegoDeEvolucion}{t: $toroide$, pos: $\ent \times \ent$}{\bool}{
        \\\IfThenElse{seMantieneViva(t, pos) \oLuego vivePorReproduccion(t, pos)}{
            \True
        }{
            \False
        }
    }
    \pred{seMantieneViva}{t: $toroide$, pos: $\ent \times \ent$}{
        estaViva(t[posicion_0][posicion_1]) \yLuego 2 \leq vivasAdyacentes(t, posicion) \leq 3
    }
    \pred{vivePorReproduccion}{t: $toroide$, pos: $\ent \times \ent$}{
        (\neg estaViva(t[posicion_0][posicion_1]) \yLuego vivasAdyacentes(t, posicion) = 3)
    }
    \aux{vivasAdyacentes}{t: $toroide$, pos: $\ent \times \ent$}{\ent}{
        \displaystyle\\(\sum_{i=-1}^1 \sum_{j=-1}^1 \IfThenElse{valorPosicionNormalizada(t, (pos_0 + i, pos_1 + j)) = \True}{ 1 }{ 0 }) 
        \\- (\IfThenElse{estaViva(t, pos_0, pos_1)}{1}{0})
    }
    \aux{valorPosicionNormalizada}{t: $toroide$, pos: $\ent \times \ent$}{\bool}{
        \\t[normalizarIndice(filas(t), pos_0)][normalizarIndice(columnas(t), pos_1)]
    \\}
    \aux{normalizarIndice}{limite: \ent, i: \ent}{\ent}{
        \IfThenElse{i < 0}{
            (i + limite)
        }{\\
            (\IfThenElse{i \geq limite}{
                (i - limite)
            }{
                i
            })
        }
    }

%====================

% ===================
% === Ejercicio 5 ===
    \pred{mismaDimension}{t1: $toroide$, t2: $toroide$}{
        filas(t1) = filas(t2) \yLuego columnas(t1) = columnas(t2)
    }

    \pred{esEvolucion}{t: $toroide$, te: $toroide$}{
        mismaDimension(t, te) \yLuego (\forall i, j : \ent)(enRango(t, i, j)\implicaLuego(te[i][j] = valorLuegoDeEvolucion(t, (i, j))))
    }
%====================

% === Ejercicio 6 ===

    \pred{esEvolucionMultiple}{t: $toroide$, te: $toroide$, k: \ent}{
        (\exists ts: \TLista{toroide})((length(ts) = k + 1) \yLuego ts[0] = t \yLuego ordenadaPorEvolucion(ts) \yLuego te = ts[k])
    }

    \pred{ordenadaPorEvolucion}{ts: $\TLista{toroide}$}{
        (\forall i: \ent)((0 < i \leq k)\implicaLuego\\esEvolucion(ts[i - 1], ts[i]))
    }

%====================

%====================
% === Ejercicio 11 ==

\pred{esVistaTrasladada}{t1: $toroide$, t2: $toroide$}{
    (\exists i, j: \ent)(
            \\(\forall x, y : \ent)(enRango(t1, x, y)\implicaLuego(t1[x][y] = valorPosicionNormalizada(t2, (x + i, y + j)))))
}

\section{Decisiones tomadas}
Intuimos que una posición tiene 8 adyacentes independientemente del tamaño del toroide, 
implicando esto que dentro de las adyacentes a una posición se pueden contar posiciones repetidas
% Sólo para decisiones de alto nivel, no para explicar la solución a los ejercicios: CONSULTAR!

\end{document}
