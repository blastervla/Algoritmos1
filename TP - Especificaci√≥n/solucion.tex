\documentclass[a4paper]{article}
\usepackage{ifthen}
\usepackage{amssymb}
\usepackage{multicol}
\usepackage{graphicx}
\usepackage[absolute]{textpos}

\usepackage[noload]{qtree}
%\usepackage{xspace,rotating,calligra,dsfont,ifthen}
\usepackage{xspace,rotating,dsfont,ifthen}
\usepackage[spanish,activeacute]{babel}
\usepackage[utf8]{inputenc}
\usepackage{pgfpages}
\usepackage{pgf,pgfarrows,pgfnodes,pgfautomata,pgfheaps,xspace,dsfont}
\usepackage{listings}
\usepackage{multicol}


\makeatletter

\@ifclassloaded{beamer}{%
  \newcommand{\tocarEspacios}{%
    \addtolength{\leftskip}{4em}%
    \addtolength{\parindent}{-3em}%
  }%
}
{%
  \usepackage[top=1cm,bottom=2cm,left=1cm,right=1cm]{geometry}%
  \usepackage{color}%
  \newcommand{\tocarEspacios}{%
    \addtolength{\leftskip}{5em}%
    \addtolength{\parindent}{-3em}%
  }%
}

\newcommand{\encabezadoDeProc}[4]{%
  % Ponemos la palabrita problema en tt
%  \noindent%
  {\normalfont\bfseries\ttfamily proc}%
  % Ponemos el nombre del problema
  \ %
  {\normalfont\ttfamily #2}%
  \
  % Ponemos los parametros
  (#3)%
  \ifthenelse{\equal{#4}{}}{}{%
  \ =\ %
  % Ponemos el nombre del resultado
  {\normalfont\ttfamily #1}%
  % Por ultimo, va el tipo del resultado
  \ : #4}
}

\newcommand{\encabezadoDeTipo}[2]{%
  % Ponemos la palabrita tipo en tt
  {\normalfont\bfseries\ttfamily tipo}%
  % Ponemos el nombre del tipo
  \ %
  {\normalfont\ttfamily #2}%
  \ifthenelse{\equal{#1}{}}{}{$\langle$#1$\rangle$}
}

% Primero definiciones de cosas al estilo title, author, date

\def\materia#1{\gdef\@materia{#1}}
\def\@materia{No especifi\'o la materia}
\def\lamateria{\@materia}

\def\cuatrimestre#1{\gdef\@cuatrimestre{#1}}
\def\@cuatrimestre{No especifi\'o el cuatrimestre}
\def\elcuatrimestre{\@cuatrimestre}

\def\anio#1{\gdef\@anio{#1}}
\def\@anio{No especifi\'o el anio}
\def\elanio{\@anio}

\def\fecha#1{\gdef\@fecha{#1}}
\def\@fecha{\today}
\def\lafecha{\@fecha}

\def\nombre#1{\gdef\@nombre{#1}}
\def\@nombre{No especific'o el nombre}
\def\elnombre{\@nombre}

\def\practicas#1{\gdef\@practica{#1}}
\def\@practica{No especifi\'o el n\'umero de pr\'actica}
\def\lapractica{\@practica}


% Esta macro convierte el numero de cuatrimestre a palabras
\newcommand{\cuatrimestreLindo}{
  \ifthenelse{\equal{\elcuatrimestre}{1}}
  {Primer cuatrimestre}
  {\ifthenelse{\equal{\elcuatrimestre}{2}}
  {Segundo cuatrimestre}
  {Verano}}
}


\newcommand{\depto}{{UBA -- Facultad de Ciencias Exactas y Naturales --
      Departamento de Computaci\'on}}

\newcommand{\titulopractica}{
  \centerline{\depto}
  \vspace{1ex}
  \centerline{{\Large\lamateria}}
  \vspace{0.5ex}
  \centerline{\cuatrimestreLindo de \elanio}
  \vspace{2ex}
  \centerline{{\huge Pr\'actica \lapractica -- \elnombre}}
  \vspace{5ex}
  \arreglarincisos
  \newcounter{ejercicio}
  \newenvironment{ejercicio}{\stepcounter{ejercicio}\textbf{Ejercicio
      \theejercicio}%
    \renewcommand\@currentlabel{\theejercicio}%
  }{\vspace{0.2cm}}
}


\newcommand{\titulotp}{
  \centerline{\depto}
  \vspace{1ex}
  \centerline{{\Large\lamateria}}
  \vspace{0.5ex}
  \centerline{\cuatrimestreLindo de \elanio}
  \vspace{0.5ex}
  \centerline{\lafecha}
  \vspace{2ex}
  \centerline{{\huge\elnombre}}
  \vspace{5ex}
}


%practicas
\newcommand{\practica}[2]{%
    \title{Pr\'actica #1 \\ #2}
    \author{Algoritmos y Estructuras de Datos I}
    \date{Primer Cuatrimestre 2018}

    \maketitlepractica{#1}{#2}
}

\newcommand \maketitlepractica[2] {%
\begin{center}
\begin{tabular}{r cr}
 \begin{tabular}{c}
{\large\bf\textsf{\ Algoritmos y Estructuras de Datos I\ }}\\
Primer Cuatrimestre 2018\\
\title{\normalsize Gu\'ia Pr\'actica #1 \\ \textbf{#2}}\\
\@title
\end{tabular} &
\begin{tabular}{@{} p{1.6cm} @{}}
\includegraphics[width=1.6cm]{logodpt.jpg}
\end{tabular} &
\begin{tabular}{l @{}}
 \emph{Departamento de Computaci\'on} \\
 \emph{Facultad de Ciencias Exactas y Naturales} \\
 \emph{Universidad de Buenos Aires} \\
\end{tabular}
\end{tabular}
\end{center}

\bigskip
}


% Simbolos varios

\newcommand{\ent}{\ensuremath{\mathds{Z}}}
\newcommand{\float}{\ensuremath{\mathds{R}}}
\newcommand{\bool}{\ensuremath{\mathsf{Bool}}}
\newcommand{\True}{\ensuremath{\mathrm{true}}}
\newcommand{\False}{\ensuremath{\mathrm{false}}}
\newcommand{\Then}{\ensuremath{\rightarrow}}
\newcommand{\Iff}{\ensuremath{\leftrightarrow}}
\newcommand{\implica}{\ensuremath{\longrightarrow}}
\newcommand{\IfThenElse}[3]{\ensuremath{\mathsf{if}\ #1\ \mathsf{then}\ #2\ \mathsf{else}\ #3\ \mathsf{fi}}}
\newcommand{\In}{\textsf{in }}
\newcommand{\Out}{\textsf{out }}
\newcommand{\Inout}{\textsf{inout }}
\newcommand{\yLuego}{\land _L}
\newcommand{\oLuego}{\lor _L}
\newcommand{\implicaLuego}{\implica _L}
\newcommand{\existe}[3]{\ensuremath{(\exists #1:\ent) \ #2 \leq #1 < #3 \ }}
\newcommand{\paraTodo}[3]{\ensuremath{(\forall #1:\ent) \ #2 \leq #1 < #3 \ }}

% Símbolo para marcar los ejercicios importantes (estrellita)
\newcommand\importante{\raisebox{0.5pt}{\ensuremath{\bigstar}}}


\newcommand{\rango}[2]{[#1\twodots#2]}
\newcommand{\comp}[2]{[\,#1\,|\,#2\,]}

\newcommand{\rangoac}[2]{(#1\twodots#2]}
\newcommand{\rangoca}[2]{[#1\twodots#2)}
\newcommand{\rangoaa}[2]{(#1\twodots#2)}

%ejercicios
\newtheorem{exercise}{Ejercicio}
\newenvironment{ejercicio}[1][]{\begin{exercise}#1\rm}{\end{exercise} \vspace{0.2cm}}
\newenvironment{items}{\begin{enumerate}[a)]}{\end{enumerate}}
\newenvironment{subitems}{\begin{enumerate}[i)]}{\end{enumerate}}
\newcommand{\sugerencia}[1]{\noindent \textbf{Sugerencia:} #1}

\lstnewenvironment{code}{
    \lstset{% general command to set parameter(s)
        language=C++, basicstyle=\small\ttfamily, keywordstyle=\slshape,
        emph=[1]{tipo,usa}, emphstyle={[1]\sffamily\bfseries},
        morekeywords={tint,forn,forsn},
        basewidth={0.47em,0.40em},
        columns=fixed, fontadjust, resetmargins, xrightmargin=5pt, xleftmargin=15pt,
        flexiblecolumns=false, tabsize=2, breaklines, breakatwhitespace=false, extendedchars=true,
        numbers=left, numberstyle=\tiny, stepnumber=1, numbersep=9pt,
        frame=l, framesep=3pt,
    }
   \csname lst@SetFirstLabel\endcsname}
  {\csname lst@SaveFirstLabel\endcsname}


%tipos basicos
\newcommand{\rea}{\ensuremath{\mathsf{Float}}}
\newcommand{\cha}{\ensuremath{\mathsf{Char}}}
\newcommand{\str}{\ensuremath{\mathsf{String}}}

\newcommand{\mcd}{\mathrm{mcd}}
\newcommand{\prm}[1]{\ensuremath{\mathsf{prm}(#1)}}
\newcommand{\sgd}[1]{\ensuremath{\mathsf{sgd}(#1)}}

\newcommand{\tuple}[2]{\ensuremath{#1 \times #2}}

%listas
\newcommand{\TLista}[1]{\ensuremath{seq \langle #1\rangle}}
\newcommand{\lvacia}{\ensuremath{[\ ]}}
\newcommand{\lv}{\ensuremath{[\ ]}}
\newcommand{\longitud}[1]{\ensuremath{|#1|}}
\newcommand{\cons}[1]{\ensuremath{\mathsf{addFirst}}(#1)}
\newcommand{\indice}[1]{\ensuremath{\mathsf{indice}}(#1)}
\newcommand{\conc}[1]{\ensuremath{\mathsf{concat}}(#1)}
\newcommand{\cab}[1]{\ensuremath{\mathsf{head}}(#1)}
\newcommand{\cola}[1]{\ensuremath{\mathsf{tail}}(#1)}
\newcommand{\sub}[1]{\ensuremath{\mathsf{subseq}}(#1)}
\newcommand{\en}[1]{\ensuremath{\mathsf{en}}(#1)}
\newcommand{\cuenta}[2]{\mathsf{cuenta}\ensuremath{(#1, #2)}}
\newcommand{\suma}[1]{\mathsf{suma}(#1)}
\newcommand{\twodots}{\ensuremath{\mathrm{..}}}
\newcommand{\masmas}{\ensuremath{++}}
\newcommand{\matriz}[1]{\TLista{\TLista{#1}}}

% Acumulador
\newcommand{\acum}[1]{\ensuremath{\mathsf{acum}}(#1)}
\newcommand{\acumselec}[3]{\ensuremath{\mathrm{acum}(#1 |  #2, #3)}}

% \selector{variable}{dominio}
\newcommand{\selector}[2]{#1~\ensuremath{\leftarrow}~#2}
\newcommand{\selec}{\ensuremath{\leftarrow}}

\newcommand{\pred}[3]{%
    {\normalfont\bfseries\ttfamily pred }%
    {\normalfont\ttfamily #1}%
    \ifthenelse{\equal{#2}{}}{}{\ (#2) }%
    \{\ensuremath{#3}\}%
    {\normalfont\bfseries\,\par}%
  }

\newenvironment{proc}[4][res]{%
  % El parametro 1 (opcional) es el nombre del resultado
  % El parametro 2 es el nombre del problema
  % El parametro 3 son los parametros
  % El parametro 4 es el tipo del resultado
  % Preambulo del ambiente problema
  % Tenemos que definir los comandos requiere, asegura, modifica y aux
  \newcommand{\pre}[2][]{%
    {\normalfont\bfseries\ttfamily Pre}%
    \ifthenelse{\equal{##1}{}}{}{\ {\normalfont\ttfamily ##1} :}\ %
    \{\ensuremath{##2}\}%
    {\normalfont\bfseries\,\par}%
  }
  \newcommand{\post}[2][]{%
    {\normalfont\bfseries\ttfamily Post}%
    \ifthenelse{\equal{##1}{}}{}{\ {\normalfont\ttfamily ##1} :}\
    \{\ensuremath{##2}\}%
    {\normalfont\bfseries\,\par}%
  }
  \renewcommand{\aux}[4]{%
    {\normalfont\bfseries\ttfamily fun\ }%
    {\normalfont\ttfamily ##1}%
    \ifthenelse{\equal{##2}{}}{}{\ (##2)}\ : ##3\, = \ensuremath{##4}%
    {\normalfont\bfseries\,;\par}%
  }
  \newcommand{\res}{#1}
  \vspace{1ex}
  \noindent
  \encabezadoDeProc{#1}{#2}{#3}{#4}
  % Abrimos la llave
  \{\par%
  \tocarEspacios
}
% Ahora viene el cierre del ambiente problema
{
  % Cerramos la llave
  \noindent\}
  \vspace{1ex}
}


  \newcommand{\aux}[4]{%
    {\normalfont\bfseries\ttfamily fun\ }%
    {\normalfont\ttfamily #1}%
    \ifthenelse{\equal{#2}{}}{}{\ (#2)}\ : #3\, = \ensuremath{#4}%
    {\normalfont\bfseries\,;\par}%
  }


% \newcommand{\pre}[1]{\textsf{pre}\ensuremath{(#1)}}

\newcommand{\procnom}[1]{\textsf{#1}}
\newcommand{\procil}[3]{\textsf{proc #1}\ensuremath{(#2) = #3}}
\newcommand{\procilsinres}[2]{\textsf{proc #1}\ensuremath{(#2)}}
\newcommand{\preil}[2]{\textsf{Pre #1: }\ensuremath{#2}}
\newcommand{\postil}[2]{\textsf{Post #1: }\ensuremath{#2}}
\newcommand{\auxil}[2]{\textsf{fun }\ensuremath{#1 = #2}}
\newcommand{\auxilc}[4]{\textsf{fun }\ensuremath{#1( #2 ): #3 = #4}}
\newcommand{\auxnom}[1]{\textsf{fun }\ensuremath{#1}}
\newcommand{\auxpred}[3]{\textsf{pred }\ensuremath{#1( #2 ) \{ #3 \}}}

\newcommand{\comentario}[1]{{/*\ #1\ */}}

\newcommand{\nom}[1]{\ensuremath{\mathsf{#1}}}


% En las practicas/parciales usamos numeros arabigos para los ejercicios.
% Aca cambiamos los enumerate comunes para que usen letras y numeros
% romanos
\newcommand{\arreglarincisos}{%
  \renewcommand{\theenumi}{\alph{enumi}}
  \renewcommand{\theenumii}{\roman{enumii}}
  \renewcommand{\labelenumi}{\theenumi)}
  \renewcommand{\labelenumii}{\theenumii)}
}



%%%%%%%%%%%%%%%%%%%%%%%%%%%%%% PARCIAL %%%%%%%%%%%%%%%%%%%%%%%%
\let\@xa\expandafter
\newcommand{\tituloparcial}{\centerline{\depto -- \lamateria}
  \centerline{\elnombre -- \lafecha}%
  \setlength{\TPHorizModule}{10mm} % Fija las unidades de textpos
  \setlength{\TPVertModule}{\TPHorizModule} % Fija las unidades de
                                % textpos
  \arreglarincisos
  \newcounter{total}% Este contador va a guardar cuantos incisos hay
                    % en el parcial. Si un ejercicio no tiene incisos,
                    % cuenta como un inciso.
  \newcounter{contgrilla} % Para hacer ciclos
  \newcounter{columnainicial} % Se van a usar para los cline cuando un
  \newcounter{columnafinal}   % ejercicio tenga incisos.
  \newcommand{\primerafila}{}
  \newcommand{\segundafila}{}
  \newcommand{\rayitas}{} % Esto va a guardar los \cline de los
                          % ejercicios con incisos, asi queda mas bonito
  \newcommand{\anchodegrilla}{20} % Es para textpos
  \newcommand{\izquierda}{7} % Estos dos le dicen a textpos donde colocar
  \newcommand{\abajo}{2}     % la grilla
  \newcommand{\anchodecasilla}{0.4cm}
  \setcounter{columnainicial}{1}
  \setcounter{total}{0}
  \newcounter{ejercicio}
  \setcounter{ejercicio}{0}
  \renewenvironment{ejercicio}[1]
  {%
    \stepcounter{ejercicio}\textbf{\noindent Ejercicio \theejercicio. [##1
      puntos]}% Formato
    \renewcommand\@currentlabel{\theejercicio}% Esto es para las
                                % referencias
    \newcommand{\invariante}[2]{%
      {\normalfont\bfseries\ttfamily invariante}%
      \ ####1\hspace{1em}####2%
    }%
    \newcommand{\Proc}[5][result]{
      \encabezadoDeProc{####1}{####2}{####3}{####4}\hspace{1em}####5}%
  }% Aca se termina el principio del ejercicio
  {% Ahora viene el final
    % Esto suma la cantidad de incisos o 1 si no hubo ninguno
    \ifthenelse{\equal{\value{enumi}}{0}}
    {\addtocounter{total}{1}}
    {\addtocounter{total}{\value{enumi}}}
    \ifthenelse{\equal{\value{ejercicio}}{1}}{}
    {
      \g@addto@macro\primerafila{&} % Si no estoy en el primer ej.
      \g@addto@macro\segundafila{&}
    }
    \ifthenelse{\equal{\value{enumi}}{0}}
    {% No tiene incisos
      \g@addto@macro\primerafila{\multicolumn{1}{|c|}}
      \bgroup% avoid overwriting somebody else's value of \tmp@a
      \protected@edef\tmp@a{\theejercicio}% expand as far as we can
      \@xa\g@addto@macro\@xa\primerafila\@xa{\tmp@a}%
      \egroup% restore old value of \tmp@a, effect of \g@addto.. is

      \stepcounter{columnainicial}
    }
    {% Tiene incisos
      % Primero ponemos el encabezado
      \g@addto@macro\primerafila{\multicolumn}% Ahora el numero de items
      \bgroup% avoid overwriting somebody else's value of \tmp@a
      \protected@edef\tmp@a{\arabic{enumi}}% expand as far as we can
      \@xa\g@addto@macro\@xa\primerafila\@xa{\tmp@a}%
      \egroup% restore old value of \tmp@a, effect of \g@addto.. is
      % global
      % Ahora el formato
      \g@addto@macro\primerafila{{|c|}}%
      % Ahora el numero de ejercicio
      \bgroup% avoid overwriting somebody else's value of \tmp@a
      \protected@edef\tmp@a{\theejercicio}% expand as far as we can
      \@xa\g@addto@macro\@xa\primerafila\@xa{\tmp@a}%
      \egroup% restore old value of \tmp@a, effect of \g@addto.. is
      % global
      % Ahora armamos la segunda fila
      \g@addto@macro\segundafila{\multicolumn{1}{|c|}{a}}%
      \setcounter{contgrilla}{1}
      \whiledo{\value{contgrilla}<\value{enumi}}
      {%
        \stepcounter{contgrilla}
        \g@addto@macro\segundafila{&\multicolumn{1}{|c|}}
        \bgroup% avoid overwriting somebody else's value of \tmp@a
        \protected@edef\tmp@a{\alph{contgrilla}}% expand as far as we can
        \@xa\g@addto@macro\@xa\segundafila\@xa{\tmp@a}%
        \egroup% restore old value of \tmp@a, effect of \g@addto.. is
        % global
      }
      % Ahora armo las rayitas
      \setcounter{columnafinal}{\value{columnainicial}}
      \addtocounter{columnafinal}{-1}
      \addtocounter{columnafinal}{\value{enumi}}
      \bgroup% avoid overwriting somebody else's value of \tmp@a
      \protected@edef\tmp@a{\noexpand\cline{%
          \thecolumnainicial-\thecolumnafinal}}%
      \@xa\g@addto@macro\@xa\rayitas\@xa{\tmp@a}%
      \egroup% restore old value of \tmp@a, effect of \g@addto.. is
      \setcounter{columnainicial}{\value{columnafinal}}
      \stepcounter{columnainicial}
    }
    \setcounter{enumi}{0}%
    \vspace{0.2cm}%
  }%
  \newcommand{\tercerafila}{}
  \newcommand{\armartercerafila}{
    \setcounter{contgrilla}{1}
    \whiledo{\value{contgrilla}<\value{total}}
    {\stepcounter{contgrilla}\g@addto@macro\tercerafila{&}}
  }
  \newcommand{\grilla}{%
    \g@addto@macro\primerafila{&\textbf{TOTAL}}
    \g@addto@macro\segundafila{&}
    \g@addto@macro\tercerafila{&}
    \armartercerafila
    \ifthenelse{\equal{\value{total}}{\value{ejercicio}}}
    {% No hubo incisos
      \begin{textblock}{\anchodegrilla}(\izquierda,\abajo)
        \begin{tabular}{|*{\value{total}}{p{\anchodecasilla}|}c|}
          \hline
          \primerafila\\
          \hline
          \tercerafila\\
          \tercerafila\\
          \hline
        \end{tabular}
      \end{textblock}
    }
    {% Hubo incisos
      \begin{textblock}{\anchodegrilla}(\izquierda,\abajo)
        \begin{tabular}{|*{\value{total}}{p{\anchodecasilla}|}c|}
          \hline
          \primerafila\\
          \rayitas
          \segundafila\\
          \hline
          \tercerafila\\
          \tercerafila\\
          \hline
        \end{tabular}
      \end{textblock}
    }
  }%
  \vspace{0.4cm}
  \textbf{Nro. de orden:}

  \textbf{LU:}

  \textbf{Apellidos:}

  \textbf{Nombres:}
  \vspace{0.5cm}
}



% AMBIENTE CONSIGNAS
% Se usa en el TP para ir agregando las cosas que tienen que resolver
% los alumnos.
% Dentro del ambiente hay que usar \item para cada consigna

\newcounter{consigna}
\setcounter{consigna}{0}

\newenvironment{consignas}{%
  \newcommand{\consigna}{\stepcounter{consigna}\textbf{\theconsigna.}}%
  \renewcommand{\ejercicio}[1]{\item ##1 }
  \renewcommand{\proc}[5][result]{\item
    \encabezadoDeProc{##1}{##2}{##3}{##4}\hspace{1em}##5}%
  \newcommand{\invariante}[2]{\item%
    {\normalfont\bfseries\ttfamily invariante}%
    \ ##1\hspace{1em}##2%
  }
  \renewcommand{\aux}[4]{\item%
    {\normalfont\bfseries\ttfamily aux\ }%
    {\normalfont\ttfamily ##1}%
    \ifthenelse{\equal{##2}{}}{}{\ (##2)}\ : ##3 \hspace{1em}##4%
  }
  % Comienza la lista de consignas
  \begin{list}{\consigna}{%
      \setlength{\itemsep}{0.5em}%
      \setlength{\parsep}{0cm}%
    }
}%
{\end{list}}



% para decidir si usar && o ^
\newcommand{\y}[0]{\ensuremath{\land}}

% macros de correctitud
\newcommand{\semanticComment}[2]{#1 \ensuremath{#2};}
\newcommand{\namedSemanticComment}[3]{#1 #2: \ensuremath{#3};}


\newcommand{\local}[1]{\semanticComment{local}{#1}}

\newcommand{\vale}[1]{\semanticComment{vale}{#1}}
\newcommand{\valeN}[2]{\namedSemanticComment{vale}{#1}{#2}}
\newcommand{\impl}[1]{\semanticComment{implica}{#1}}
\newcommand{\implN}[2]{\namedSemanticComment{implica}{#1}{#2}}
\newcommand{\estado}[1]{\semanticComment{estado}{#1}}

\newcommand{\invarianteCN}[2]{\namedSemanticComment{invariante}{#1}{#2}}
\newcommand{\invarianteC}[1]{\semanticComment{invariante}{#1}}
\newcommand{\varianteCN}[2]{\namedSemanticComment{variante}{#1}{#2}}
\newcommand{\varianteC}[1]{\semanticComment{variante}{#1}}


\usepackage{a4wide}
\usepackage{amsmath, amscd, amssymb, amsthm, latexsym}
\usepackage[spanish,activeacute]{babel}
\usepackage{enumerate}

\setlength{\parskip}{0.1em}
\usepackage{caratula} % Version modificada para usar las macros de algo1 de ~> https://github.com/bcardiff/dc-tex

\begin{document}

\titulo{TP de Especificación}
\subtitulo{Juego de la vida toroidal}
\fecha{\today}
\materia{Algoritmos y Estructuras de Datos I}
\grupo{Grupo: Java the Hutt$;$}

\newcommand{\senial}{\textit{se\~nal}}

% Pongan cuantos integrantes quieran
\integrante{Pomsztein, Vladimir}{364/18}{blastervla@gmail.com}
\integrante{Zinik, Luciano}{290/17}{lzinik@gmail.com}

\maketitle

\section{Problemas}

\subsection{esValido}
% 1. Dado un toroide verifica si es válido.
\begin{proc}{esValido}{\In t: $toroide$, \Out result: $\bool$}{}
    \pre{\True}
    % POST: - Tiene la misma cantidad de columnas en cada fila
    %       - No es vacía
    \post{$result = \True $ \Iff esToroideValido(t)}
\end{proc}

\subsection{posicionesVivas}

% 2. Dado un toroide devuelva todas las posiciones vivas.
\begin{proc}{posicionesVivas}{\In t: $toroide$, \Out vivas: $\TLista{\ent \times \ent}$}{}
    \pre{esToroideValido(t)}
    % La posición pertenece a result sí y sólo sí está viva
    \post{(\forall i, j : \ent)(enRango(t, i, j) \implicaLuego ((i, j) \in vivas \Iff estaViva(t[i][j])))}
\end{proc}

\subsection{densidadPoblacion}
% 3. Dado un toroide devuelva su densidad de población, es decir, la relación entre 
% la cantidad de posiciones vivas y la cantidad total de posiciones
\begin{proc}{densidadPoblacion}{\In t: $toroide$, \Out result: $\float$}{}
    \pre{esToroideValido(t)}
    % densidad = cantidad vivas / cantidad total
    \post{result = cantidadVivas(t) / cantidadTotal(t)}
    \aux{cantidadTotal}{t: $toroide$}{\ent}{
        filas(t) \times columnas(t)
    }
\end{proc}

\subsection{evolucionDePosicion}
% 4. Dado un toroide y una posición del mismo, indique si dicha posición estaría viva 
% luego de un tick.
\begin{proc}{evolucionDePosicion}{\In t: $toroide$, \In posicion: $\ent \times \ent$, \Out result: $\bool$}{}
    % la posicion tiene que estar en rango
    \pre{esToroideValido(t) \yLuego enRango(t, posicion_0, posicion_1)}
    % Evolución: 
    %   - Viva con < 2 adyacentes vivas --> muere
    %   - Viva con 2 o 3 adyacentes vivas --> vive
    %   - Viva con > 3 adyacentes vivas --> muere
    %   - Muerta con 3 adyacentes vivas --> vive
    \post{result = valorLuegoDeEvolucion(t, posicion)} % estaViva y tiene 2 o 3 adyacentes o ¬estaViva y tiene 3 adyacentes
    
\end{proc}

\subsection{evolucionToroide}
% 5. Dado un toroide lo evoluciona un tick.
\begin{proc}{evolucionToroide}{\Inout t: $toroide$}{}
    \pre{esToroideValido(t) \yLuego t = T0}
    % Quizás usando una sumatoria que recorra todo el toroide (actually, son 2)
    % Y que en cada término de la sumatoria cambie el estado de esa posicion
    % Alternativa (me gusta más): Definimios la metavariable t1, t = t2 sí y sólo sí cada 
    % posición de t2 es igual a dicha posición de t1 pero evolucionada
    \post{esEvolucion(T0, t)}
\end{proc}

\subsection{evolucionMultiple}
% 6. Dado un toroide t y un natural k, devuelve el toroide resultante de evolucionar 
% t por k ticks
\begin{proc}{evolucionMultiple}{\In t: $toroide$, \In k: $\ent$, \Out result: $toroide$}{}
    % k debe ser natural
    \pre{esToroideValido(t) \land k \geq 0}
    % result = t1 <=> LA SUMATORIA da 0
    % LA SUMATORIA da 0 <=> la evolución de t = evolucion de t1 (se puede porque
    % evolucionToroide es inout, si es distinta suma 1)

    % se puede definir un aux recursivo que checkee la evolución por ticks hasta que sea 0
    % y que devuelva un booleano si es igual
    \post{ esEvolucionMultiple(t, result, k) }
\end{proc}

\subsection{esPeriodico\\}
% 7. Dado un toroide devuelve si el mismo es periódico o no. En caso de serlo, 
% se debe devolver en p la mínima cantidad de ticks en la cual se repite el patrón. 
% Decimos que un toroide es periódico si pasada cierta cantidad de ticks, vuelve a tener
% exactamente la misma configuración que tenía originalmente.
\begin{proc}{esPeriodico}{\In t: $toroide$, \Inout p: $\ent$, \Out result: $\bool$}{}
    \pre{esToroideValido(t)}

    \post{result = \True \Iff (\exists k: \ent)(esEvolucionMultiple(t, t, k) 
    \breakAndSpace{1.3em}\yLuego esElMenorKAntesDeEvolucion(t, t, k)) \yLuego p = k}
\end{proc}

\subsection{primosLejanos}
% 8. Dados dos toroides, devuelve si uno es la evolución múltiple del otro.
\begin{proc}{primosLejanos}{\In t1: $toroide$, \In t2: $toroide$, \Out primos: $\bool$}{}
    \pre{esToroideValido(t1) \land esToroideValido(t2) \yLuego mismaDimension(t1, t2)}
    \post{primos = \True \Iff 
    ((\exists k: \ent)((k > 0)\breakAndSpace{1.3em}\yLuego(
        (esEvolucionMultiple(t1, t2, k))
        \oLuego
        (esEvolucionMultiple(t2, t1, k))
        )))}
\end{proc}

\subsection{seleccionNatural}
% 9. Dada una secuencia de toroides, devuelve el índice de aquel toroide que más ticks 
% tardará en morir. Se considera que un toroide muere cuando no tiene posiciones vivas.
\begin{proc}{seleccionNatural}{\In ts: $\TLista{toroide}$, \Out res: $\ent$}{}
    \pre{length(ts) > 0 \yLuego (todosToroidesValidos(ts) \yLuego algunToroideMuere(ts))}

    \post{(0 \leq res < length(ts)) \yLuego tieneElMayorTiempoDeMuerte(ts[res], ts)}

    \pred{todosToroidesValidos}{ts: $\TLista{toroide}$}{
        (\forall t: $toroide$)((t \in ts) \implicaLuego esToroideValido(t))
    }

    \pred{algunToroideMuere}{ts: $\TLista{toroide}$}{
        (\exists t: $toroide$)((t \in ts)\yLuego(muere(t)))
    }

    \pred{tieneElMayorTiempoDeMuerte}{t: $toroide$, ts: $\TLista{toroide}$}{
        (\forall tx: toroide)((tx \in ts)\breakAndSpace{1.3em}\implicaLuego(cantidadDeTicksHastaMuerte(t) \geq cantidadDeTicksHastaMuerte(tx)))
    }

    \aux{cantidadDeTicksHastaMuerte}{t: $toroide$}{\ent} {
        \breakAndSpace{1.3em}\IfThenElse{(\exists tx: toroide)(mismaDimension(t, tx) \yLuego cantidadVivas(tx) = 0)\breakAndSpace{1.3em}\yLuego(\exists k: \ent)(k > 0 \yLuego esEvolucionMultiple(t, tx, k) \yLuego esElMenorKAntesDeEvolucion(t, tx, k))\breakAndSpace{1em}}{
            k
        }{
            -1
        }
    }

    \pred{muere}{t: $toroide$}{
        cantidadDeTicksHastaMuerte(t) \neq -1
    }
\end{proc}

\subsection{fusionar}
% 10. Dados dos toroides de la misma dimensión, devuelva otro (de la misma dimensión) 
% que tenga vivas solo aquellas posiciones que estaban vivas en ambos toroides
\begin{proc}{fusionar}{\In t1: $toroide$, \In t2: $toroide$, \Out res: $toroide$}{}
    \pre{(esToroideValido(t1) \land esToroideValido(t2)) \yLuego mismaDimension(t1, t2)}
    % res = tf <=> para todo i,j estaViva(tf[i][j]) 
    % <=> estaViva(t1[i][j]) y estaViva(t2[i][j])
    \post{mismaDimension(res, t1)\yLuego\breakAndSpace{1.3em}(\forall i, j: \ent)((enRango(res, i, j))\implicaLuego(estaViva(res, i, j) \breakAndSpace{1.1em}\Iff estaViva(t1, i, j) \land estaViva(t2, i, j)))}
\end{proc}

\subsection{vistaTrasladada}
% 11. Dados dos toroides de la misma dimensión, indica si uno es el resultado de trasladar
% la vista en el otro. Es decir, que moviendo el centro del eje de coordenadas de uno 
% de los toroides en alguna dirección, se obtiene el otro.
\begin{proc}{vistaTrasladada}{\In t1: $toroide$, \In t2: $toroide$, \Out res: $\bool$}{}
    \pre{(esToroideValido(t1) \land esToroideValido(t2)) \yLuego mismaDimension(t1, t2)}
    \post{res = \True \Iff esVistaTrasladada(t1, t2)}
\end{proc}

\subsection{enCrecimiento}
% 12. Verifica si la menor superficie que cubre a todas las celdas vivas se va incrementar 
% en el próximo tick
\begin{proc}{enCrecimiento}{\In t: $toroide$, \Out res: $\bool$}{}
    \pre{esToroideValido(t)}
    \post{res = \True \Iff (\exists te: toroide)(esEvolucion(t, te)\yLuego crecioSuperficie(t, te))}

    \pred{crecioSuperficie}{t: $toroide$, te: $toroide$}{
        (\exists s, se: \ent)((esMenorSuperficieDeToroide(t, s) \land esMenorSuperficieDeToroide(te, se))\breakAndSpace{1em}\yLuego s < se)
    }

    \pred{esMenorSuperficieDeToroide}{t: $toroide$, s: \ent}{
        % \\(\exists ts: \TLista{$toroide$})((compuestoPorVistasTrasladadas(ts, t) \y ordenadoSuperficieAscendente(ts)) \\\yLuego s = ts[0])
        (\exists x: toroide)((esVistaTrasladada(t, x) \yLuego tieneLaMenorSuperficieDelToroide(x)))
    }

    \pred{tieneLaMenorSuperficieDelToroide}{t: $toroide$}{
        (\forall x: toroide)(((x \neq t) \land esVistaTrasladada(t, x)) \implicaLuego esSuperficieMayorOIgual(x, t))
    }

    % \pred{compuestoPorVistasTrasladadas}{ts: \TLista{$toroide$}, t: $toroide$}{
    %     (\forall tx: $toroide$)\\((tx \in ts)\implicaLuego esVistaTrasladada(tx, t))
    % }
    % \pred{ordenadoSuperficieAscendente}{ts: \TLista{$toroide$}}{
    %     (\forall i: \ent)((0 < i < length(ts))\\\implicaLuego(esSuperficieMayorOIgual(ts[i], ts[i - 1]))))
    % }
    \pred{esSuperficieMayorOIgual}{t1: $toroide$, t2: $toroide$}{
        (\exists s1, s2: \ent)((esSuperficie(t1, s1) \land esSuperficie(t2, s2))\yLuego s1 \geq s2)
    }
    \pred{esSuperficie}{t: $toroide$, s: \ent}{
        % Tomar bolsa de superficies posibles y tomar la menor que tenga la misma cantidad
        % de celdas vivas que nuestro toroide
        % Tendríamos (xStart, yStart), (xEnd, yEnd) : Z x Z, de manera que contando las celdas
        % vivas desde xStart a xEnd columnas en cada fila desde yStart a yEnd tiene la 
        % misma cantidad que nuestro toroide original
        % (\exists rs: \TLista{(\ent \times \ent)\times(\ent \times \ent)})
        % \\(mantienenCantidadVivas(rs, t)\yLuego sonRectangulosDeToroide(rs, t)\\\yLuego ordenadoPorAreaAscendente(rs) \yLuego s = area(rs[0]))
        (\exists rect: (\ent \times \ent)\times(\ent \times \ent))(esRectanguloDeToroide(rect, t) \yLuego mantieneCantidadVivas(rect, t) \breakAndSpace{1em}\yLuego esElMenorRectangulo(rect, t) \yLuego area(rect) = s)
    }
    \pred{mantieneCantidadVivas}{rect: $(\ent \times \ent)\times(\ent \times \ent)$, t: $toroide$}{
        /* rect = (xStart, yStart) \times (xEnd, yEnd) */\breakAndSpace{1em}
        contarVivasEnArea(t, rect) = cantidadVivas(t)
    }
    \pred{esRectanguloDeToroide}{rect: $(\ent \times \ent)\times(\ent \times \ent)$, t: $toroide$}{
        /* rect = (xStart, yStart) \times (xEnd, yEnd) */\breakAndSpace{1em}
        (0 \leq rect_{0_0} \leq rect_{1_0} \leq columnas(t))\land(0 \leq rect_{0_1} \leq rect_{1_1} \leq filas(t))
    }
    \pred{esElMenorRectangulo}{r1: $(\ent \times \ent)\times(\ent \times \ent)$, t: $toroide$}{
        (\forall r2: (\ent \times \ent)\times(\ent \times \ent))((r2 \neq r1) \breakAndSpace{1em}\implicaLuego (\neg mantieneCantidadVivas(r2, t) \oLuego area(r2) \geq area(r1)))
    }
    % \pred{ordenadoPorAreaAscendente}{rs: $\TLista{(\ent \times \ent)\times(\ent \times \ent)}$}{
    %     ((\forall i: \ent)((0 < i < length(rs))\implicaLuego(area(rs[i]) \geq area(rs[i - 1]))))
    % }
    $/* rect = (xStart, yStart) \times (xEnd, yEnd) */$\breakAndSpace{-1.9em}
    \aux{contarVivasEnArea}{t: $toroide$, rect: $(\ent \times \ent)\times(\ent \times \ent)$}{\ent}{
        \breakAndSpace{1em}\displaystyle \sum_{i = rect_{0_1}}^{rect_{1_1}}(\sum_{j = rect_{0_0}}^{rect_{1_0}} \IfThenElse{estaViva(t[i][j])}{1}{0})
    }
    \aux{area}{rect: $(\ent \times \ent)\times(\ent \times \ent)$}{\ent}{
        \breakAndSpace{1em}/* base \times altura */
        \breakAndSpace{1em}(rect_{1_0} - rect_{0_0}) \times (rect_{1_1} - rect_{0_1})
    }
\end{proc}

\section{Predicados y Auxiliares generales}

% === Ejercicio 1 ===
    \gPred{noEsVacia}{t: $toroide$}{
        $(length(t) \textgreater 0)$ \yLuego (\forall x: \TLista{\bool})((x \in t) \implicaLuego ($length(x) \textgreater 0$))
        }
    \gPred{esMatriz}{t: $toroide$}{
    (\forall x, y: \TLista{\bool})((x, y \in t) \implicaLuego ($length(x) = length(y)$))
    }
    \gPred{esToroideValido}{t: $toroide$}{
        (noEsVacia(t) \land esMatriz(t))
    }
    
    \gPred{filas}{t: $toroide$}{
        $length(t)$
    }
    \gPred{columnas}{t: $toroide$}{
        \IfThenElse{$filas(t) \textgreater 0$}{$length(t[0])$}{0}
    }
% ===================
% === Ejercicio 2 ===
    \gPred{estaViva}{x: \bool}{
        x = \True
    }
    \gPred{enRango}{t: $toroide$, i: \ent, j: \ent}{
        (0 \leq i < filas(t)) \yLuego (0 \leq j < columnas(t))
    }

%====================

% === Ejercicio 3 ===
    \aux{cantidadVivas}{t: $toroide$}{\ent}{
        \displaystyle \sum_{i=0}^{filas(t)-1}(\sum_{j=0}^{columnas(t)-1} \IfThenElse{estaViva(t[i][j])}{1}{0})
    }

% ===================
% === Ejercicio 4 ===
    \aux{valorLuegoDeEvolucion}{t: $toroide$, pos: $\ent \times \ent$}{\bool}{
        \breakAndSpace{4em}\IfThenElse{seMantieneViva(t, pos) \oLuego vivePorReproduccion(t, pos)}{
            \True
        }{
            \False
        }
    }
    \gPred{seMantieneViva}{t: $toroide$, pos: $\ent \times \ent$}{
        estaViva(t[posicion_0][posicion_1]) \yLuego 2 \leq vivasAdyacentes(t, posicion) \leq 3
    }
    \gPred{vivePorReproduccion}{t: $toroide$, pos: $\ent \times \ent$}{
        (\neg estaViva(t[posicion_0][posicion_1]) \yLuego vivasAdyacentes(t, posicion) = 3)
    }
    \aux{vivasAdyacentes}{t: $toroide$, pos: $\ent \times \ent$}{\ent}{
        \displaystyle\breakAndSpace{4em}(\sum_{i=-1}^1 \sum_{j=-1}^1 \IfThenElse{valorPosicionNormalizada(t, (pos_0 + i, pos_1 + j)) = \True}{ 1 }{ 0 }) 
        \breakAndSpace{4em}- (\IfThenElse{estaViva(t, pos_0, pos_1)}{1}{0})
    }
    \aux{valorPosicionNormalizada}{t: $toroide$, pos: $\ent \times \ent$}{\bool}{
        \breakAndSpace{4em}t[normalizarIndice(filas(t), pos_0)][normalizarIndice(columnas(t), pos_1)]
    }
    \aux{normalizarIndice}{limite: \ent, i: \ent}{\ent}{
        \IfThenElse{i < 0}{
            (i + limite)
        }{\breakAndSpace{4em}
            (\IfThenElse{i \geq limite}{
                (i - limite)
            }{
                i
            })
        }
    }

%====================

% ===================
% === Ejercicio 5 ===
    \gPred{mismaDimension}{t1: $toroide$, t2: $toroide$}{
        filas(t1) = filas(t2) \yLuego columnas(t1) = columnas(t2)
    }

    \gPred{esEvolucion}{t: $toroide$, te: $toroide$}{
        mismaDimension(t, te) \yLuego (\forall i, j : \ent)(enRango(t, i, j)\implicaLuego(te[i][j] = valorLuegoDeEvolucion(t, (i, j))))
    }
%====================

%====================
% === Ejercicio 6 ===

    \gPred{esEvolucionMultiple}{t: $toroide$, te: $toroide$, k: \ent}{
        (\exists ts: \TLista{toroide})((length(ts) = k + 1) \yLuego ts[0] = t \yLuego ordenadaPorEvolucion(ts) \breakAndSpace{4em}\yLuego te = ts[k])
    }

    \gPred{ordenadaPorEvolucion}{ts: $\TLista{toroide}$}{
        (\forall i: \ent)((0 < i \leq k)\implicaLuego esEvolucion(ts[i - 1], ts[i]))
    }

%====================

%====================
% === Ejercicio 7 ===

    /* Indica si el toroide $t$ llega a tener la forma de $te$ por primera vez en el k-ésimo tick*/\breakAndSpace{1.2em}
    \gPred{esElMenorKAntesDeEvolucion}{t: $toroide$, te: $toroide$, k: \ent}{
        (\forall i: \ent)((0 < i < k) \implicaLuego \neg esEvolucionMultiple(t, te, i))
    }

%====================

%====================
% === Ejercicio 11 ==

    \gPred{esVistaTrasladada}{t1: $toroide$, t2: $toroide$}{
        (\exists i, j: \ent)((\forall x, y : \ent)(enRango(t1, x, y)\breakAndSpace{4em}\implicaLuego(t1[x][y] = valorPosicionNormalizada(t2, (x + i, y + j)))))
    }

\section{Decisiones tomadas}
Intuimos que una posición tiene 8 adyacentes independientemente del tamaño del toroide, 
implicando esto que dentro de las adyacentes a una posición se pueden contar posiciones repetidas.

\end{document}
